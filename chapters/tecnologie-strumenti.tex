\chapter{Tecnologie e strumenti}
\label{cap:tecnologie-strumenti}

\intro{In questo capitolo sono pesentate le tecnologie e gli strumenti utilizzati per lavorare al progetto di stage. Per ogni tecnologia e strumento verrà riportata la versione utilizzata, il suo scopo all'interno del progetto e una breve descrizione.}

\section{Tecnologie}

\subsection{Frontend}

\subsubsection{Xamarin.Forms}

\begin{figure}[H]
    \centering 
    \includegraphics[width=0.4\columnwidth]{images/loghi/xamarin-logo.png} 
    \caption{Logo di Xamarin.Forms}
\end{figure}

\begin{itemize}
    \item \textbf{Versione}: 5.0.0.2578;
    \item \textbf{Utilizzo}: sviluppo dell'interfaccia grafica.
\end{itemize}
Xamarin.Forms è un framework per lo sviluppo di interfacce grafiche ideato da Microsoft. È un progetto \emph{open-source} basato sull'utilizzo di XAML per per la creazione diretta delle interfacce, e C\# per il \glox{code-behind}. Questo framework è stato sviluppato per poter realizzare applicazioni Android, iOS e Windows a partire dalla stessa \emph{codebase}.\\
A partire da maggio 2024 Xamarin.Forms non è più supportato, a favore di .NET MAUI. La migrazione da Xamarin.Forms a .NET MAUI è stata considerata, però, un lavoro troppo impegnativo per il progetto di stage, che si è svolto dunque utilizzando il codice Xamarin.Forms già presente nel repository.

% \subsubsection{XAML}

% \begin{figure}[H]
%     \centering 
%     \includegraphics[width=0.2\columnwidth]{images/loghi/xaml.png} 
%     \caption{Logo di XAML}
% \end{figure}

% \begin{itemize}
%     \item \textbf{Versione}: 
%     \item \textbf{Utilizzo}: 
% \end{itemize}

%TODO

\subsection{Backend}

\subsubsection{C\#}

\begin{figure}[H]
    \centering 
    \includegraphics[width=0.15\columnwidth]{images/loghi/C_sharp-logo.png} 
    \caption{Logo di C\#}
\end{figure}

\begin{itemize}
    \item \textbf{Versione}: 12.0;
    \item \textbf{Utilizzo}: implementazione delle \glox{api} e scrittura del \glox{code-behind}.
\end{itemize}
C\# (dove il simbolo "\#" è pronunciato \emph{sharp}) è un linguaggio di programmazione sviluppato da Microsoft all'inizio degli anni 2000 per essere utilizzato con .NET Framework. È orientato agli oggetti ma supporta diversi paradigmi, tra cui funzionale e concorrente. Gode di tipizzazione forte, \emph{garbage collector} e una sintassi molto simile ai suoi predecessori e ispiratori Java e C++.

\subsubsection{.NET Standard e .NET Core}

\begin{figure}[H]
    \centering 
    \includegraphics[width=0.2\columnwidth]{images/loghi/Microsoft_.NET_logo.png} 
    \caption{Logo di .NET}
\end{figure}

\begin{itemize}
    \item \textbf{Versione}: 2.0;
    \item \textbf{Utilizzo}: implementazione delle \glox{api} e \glox{code-behind}.
\end{itemize}

\textbf{TODO}

\section{Strumenti}

\subsection{Strumenti di sviluppo}

\subsubsection{Microsoft Visual Studio}

\begin{figure}[H]
    \centering 
    \includegraphics[width=0.3\columnwidth]{images/loghi/Visual_Studio.png} 
    \caption{Logo di Visual Studio}
\end{figure}

\begin{itemize}
    \item \textbf{Versione}: 17.10.3;
    \item \textbf{Utilizzo}: codifica.
\end{itemize}
Visual Studio è un IDE sviluppato da Microsoft utilizzato principalmente per lo sviluppo con linguaggi quali C\#, C++, Java e i framework .NET e Xamarin.\\
Può essere configurato per lo sviluppo mobile offrendo di conseguenza un ampio set di funzionalità specifiche, ad esempio l'esecuzione e il debug su un simulatore, il deploy e il debug su dispositivo fisico, l'hot reload e la configurazione di diversi dispositivi su cui poter testare l'applicazione. Oltre a ciò, Visual Studio ha una funzione di correzione della sintassi e di suggerimento automatico.

\subsubsection{Microsoft SQL Server Management Studio}

\begin{figure}[H]
    \centering 
    \includegraphics[width=0.3\columnwidth]{images/loghi/sql_server.png} 
    \caption{Logo di Microsoft SQL Server Management Studio}
\end{figure}

\begin{itemize}
    \item \textbf{Versione}: 20.1.10.0;
    \item \textbf{Utilizzo}: gestione e interrogazione di database.
\end{itemize}
Microsoft SQL Server Management Studio (SSMS) è un'applicazione Microsoft concepita per l'amministrazione di server SQL Microsoft. Offre la possibilità di connettersi a server SQL e visualizzarne i contenuti grazie alla funzionalità di esplorazione delle risorse. In questo modo è possibile visualizzare i database presenti in un server e tutte le relative tabelle, eseguire query per visualizzare determinati dati oppure modificare la struttura o i contenuti delle tabelle.

\subsubsection{Swagger-UI}

\begin{figure}[H]
    \centering 
    \includegraphics[width=0.4\columnwidth]{images/loghi/swagger.png} 
    \caption{Logo di Swagger-UI}
\end{figure}

\begin{itemize}
    \item \textbf{Versione}: 3.1.0;
    \item \textbf{Utilizzo}: testing delle API.
\end{itemize}
Swagger-UI è uno strumento che offre un'interfaccia grafica per utilizzare le API all'interno di un progetto, senza necessariamente avere il codice dell'applicazione che le utilizza. In questo strumento, le API che richiedono parametri in input forniscono già il template dell'informazione in formato JSON, e, sempre nello stesso formato, restituiscono le risposte, che possono essere poi copiate o scaricate in un file dedicato.\\
All'interno del progetto di stage questo strumento è stato particolarmente utile durante la configurazione del progetto per l'esecuzione in locale e successivamente per verificare il corretto funzionamento delle nuove API sviluppate.


% \subsubsection{adb/Android SDK}

% \begin{figure}[H]
%     \centering 
%     \includegraphics[width=0.3\columnwidth]{images/loghi/} 
%     \caption{Logo di }
% \end{figure}

% \begin{itemize}
%     \item \textbf{Versione}: 1.0.41;
%     \item \textbf{Utilizzo}: 
% \end{itemize}


\subsection{Strumenti di analisi di rete}

\subsubsection{Wireshark}

\begin{figure}[H]
    \centering 
    \includegraphics[width=0.17\columnwidth]{images/loghi/wireshark.png} 
    \caption{Logo di Wireshark}
\end{figure}

\begin{itemize}
    \item \textbf{Versione}: 4.2.6;
    \item \textbf{Utilizzo}: \emph{sniffing} del traffico di rete in \emph{localhost}.
\end{itemize}
Wireshark è un software \emph{open-source} e \emph{cross-platform} utilizzato per l'analisi della comunicazione all'interno di una rete e lo sviluppo di protocolli di comunicazione. Nella pagina principale permette la selezione della rete di cui si vuole analizzare il traffico, e, iniziata l'analisi, viene mostrata la lista di pacchetti di rete trasmessi, ognuno colorato in modo diverso secondo delle regole definite in automatico. Cliccando su un pacchetto è possibile poi vederne il contenuto.\\
All'interno del progetto Wireshark è stato particolarmente utile nell'analizzare il traffico di rete in \emph{localhost} per verificare la sicurezza della trasmissione delle credenziali di autenticazione dell'app.


\subsection{Strumenti di collaborazione e gestione di progetto}

\subsubsection{Strumenti di Microsoft Office}

\begin{figure}[H]
    \begin{minipage}[b]{0.45\textwidth}
    \centering
    \includegraphics[width=.5\textwidth]{images/loghi/outlook-logo.png}
    \caption{Logo di Microsoft Outlook 2013}
    \end{minipage}
    \hfill
    \begin{minipage}[b]{0.45\textwidth}
    \centering
    \includegraphics[width=.5\textwidth]{images/loghi/Microsoft_Word.png}
    \caption{Logo di Microsoft Word 2013}
    \end{minipage}
\end{figure}

Due software Microsoft molto utilizzati da VISIONEIMPRESA sono Outlook e Word. Il primo viene utilizzato per comunicare all'interno dell'azienda. Per comunicazioni più rapide e urgenti utilizzano l'applicazione 3CX, ma per comunicazioni generali o per comunicare con gli stagisti viene utilizzato Outlook, all'interno del quale è stato creato un gruppo per facilitare l'invio di comunicazioni utili a tutti i dipendenti.\\
I documenti vengono invece scritti con Microsoft Word, e, se necessario, condivisi all'interno di un cloud aziendale chiamato Travaso, dove ogni dipendente ha una propria cartella personale. Tramite Word sono stati condivisi il piano di lavoro, il piano di test e ho redatto la documentazione riguardante il lavoro svolto su moviEXPENSE.


\subsubsection{Git}

\begin{figure}[H]
    \centering 
    \includegraphics[width=0.2\columnwidth]{images/loghi/Git.png} 
    \caption{Logo di Git}
\end{figure}

Git è un software per il controllo di versione (VCS) ideato da Linus Torvalds. È un VCS distribuito, di conseguenza non esiste un database centrale per la raccolta delle versioni, ma questo è distribuito ad ogni sviluppatore che sta lavorando al progetto.\\
Questa tipologia di software permette di organizzare i flussi di lavoro in \emph{branch}, facilitando quindi la collaborazione, e consente di raggruppare le modifiche effettuate al codice sorgente in \emph{commit} a cui viene associato un messaggio che descrive l'obiettivo di tali modifiche e un codice identificativo creato da Git; in questo modo è più semplice identificare le modifiche effettuate e, in caso sia necessario, riportare il software ad uno stato ben preciso.\\
I VCS distribuiti sono spesso utilizzati in combinazione con un servizio di hosting per il codice, ad esempio GitHub, GitLab o Bitbucket. Quest'ultimo è stato utilizzato durante lo stage ed è descritto di seguito.


\subsubsection{Suite Atlassian}

Oltre agli strumenti di Microsoft Office citati in precedenza, l'azienda fa un forte utilizzo della suite Atlassian, in particolare di Jira, Bitbucket e Confluence. Durante lo stage sono stati utilizzati questi ultimi due.

\vspace{6mm}

\begin{figure}[H]
    \centering 
    \includegraphics[width=0.5\columnwidth]{images/loghi/bitbucket.png} 
    \caption{Logo di Bitbucket}
\end{figure}

\paragraph{Bitbucket} Servizio di hosting di repository basato su Git. Permette di caricare i progetti su cui si sta lavorando contestualmente alla cronologia dei \emph{commit} di Git, condividendo così il lavoro svolto con tutti gli altri sviluppatori. Al suo interno è possibile creare \emph{branch}, visualizzare le modifiche effettuate in un preciso \emph{commit}.

\vspace{6mm}

\begin{figure}[H]
    \centering 
    \includegraphics[width=0.5\columnwidth]{images/loghi/confluence.png} 
    \caption{Logo di Confluence}
\end{figure}

\paragraph{Confluence} Software per la collaborazione online riguardo la stesura di documentazione. Viene utilizzato per scrivere la documentazione dei progetti software a cui si sta lavorando, offrendo un editor di testo integrato oppure permettendo di caricare documenti presenti in locale.


