\chapter{Tecnologie e strumenti}
\label{cap:tecnologie-strumenti}

\intro{In questo capitolo sono presentate le tecnologie e gli strumenti utilizzati per lavorare al progetto di \textit{stage}. Per ogni tecnologia e strumento verrà riportata la versione utilizzata, il suo scopo all'interno del progetto e una breve descrizione.}

\section{Tecnologie}

\subsection{\textit{Frontend}}

\subsubsection{Xamarin}

\begin{figure}[H]
    \centering 
    \includegraphics[width=0.4\columnwidth]{images/loghi/xamarin-logo.png} 
    \caption{Logo di Xamarin}
\end{figure}

\begin{itemize}
    \item \textbf{Versione}: 5.0.0.2578;
    \item \textbf{Utilizzo}: sviluppo dell'interfaccia grafica.
\end{itemize}

\noindent \textbf{Xamarin} è un \textit{framework} per lo sviluppo di applicazioni ideato da Microsoft. È un progetto \textit{open-source} basato sull'utilizzo del linguaggio \textbf{XAML} per la creazione diretta delle interfacce, e \textbf{C\#} per il \textit{\glox{code-behind}}. Questo \textit{framework} è stato sviluppato per poter realizzare applicazioni Android, iOS e Windows partendo dalla stessa \textit{\glox{codebase}}.\\
A partire da maggio 2024 Xamarin non è più supportato, a favore di .NET MAUI. La migrazione da Xamarin a .NET MAUI è stata considerata, però, un lavoro troppo impegnativo per il progetto di \textit{stage}, che si è svolto dunque utilizzando il codice Xamarin già presente nel \textit{repository}.

\subsection{\textit{Backend}}

\subsubsection{C\#}

\begin{figure}[H]
    \centering 
    \includegraphics[width=0.15\columnwidth]{images/loghi/C_sharp-logo.png} 
    \caption{Logo di C\#}
\end{figure}

\begin{itemize}
    \item \textbf{Versione}: 12.0;
    \item \textbf{Utilizzo}: implementazione delle \acrog{api} e scrittura del \textit{\glox{code-behind}}.
\end{itemize}

\noindent \textbf{C\#} (dove il simbolo "\#" è pronunciato \textit{sharp}) è un linguaggio di programmazione sviluppato da Microsoft all'inizio degli anni 2000 per essere utilizzato con .NET Framework. È orientato agli oggetti ma supporta diversi paradigmi, tra cui funzionale e concorrente. Gode di tipizzazione forte, \textit{garbage collector}, una sintassi molto simile ai suoi predecessori e ispiratori Java e C++ e utilizza l'indentazione Allman: le parentesi graffe che delimitano un blocco di codice sono allineate all'inizio della parola chiave che definisce il blocco di codice, ad esempio
\begin{minted}{csharp}
    while (condition())
    {
        do_something();
    }
\end{minted}

\subsubsection{.NET}

\begin{figure}[H]
    \centering 
    \includegraphics[width=0.17\columnwidth]{images/loghi/Microsoft_.NET_logo.png} 
    \caption{Logo di .NET}
\end{figure}
\begin{itemize}
    \item \textbf{Versione}: 8.0.302;
    \item \textbf{Utilizzo}: implementazione delle \acrog{api} e \textit{\glox{code-behind}}.
\end{itemize}

\noindent \textbf{.NET} (pronunciato \textit{dotnet}) è un \textit{framework} \textit{open-source} di Microsoft utilizzato per lo sviluppo di applicazioni. È dotato di un ecosistema che comprende:
\begin{itemize}
    \item linguaggi, tra cui C\#, F\#, Visual Basic;
    \item \textit{Common Language Runtime}, ovvero una macchina virtuale che esegue il codice intermedio ottenuto dalla compilazione dei linguaggi utilizzati;
    \item una serie di \glox{libreria} che implementano diverse funzioni, utilizzabili includendo nel codice i corretti \textit{namespace}.
\end{itemize}

\section{Strumenti}

\subsection{Strumenti di sviluppo}

\subsubsection{Microsoft Visual Studio}

\begin{figure}[H]
    \centering 
    \includegraphics[width=0.2\columnwidth]{images/loghi/Visual_Studio.png} 
    \caption{Logo di Visual Studio}
\end{figure}

\begin{itemize}
    \item \textbf{Versione}: 17.10.3;
    \item \textbf{Utilizzo}: codifica e verifica.
\end{itemize}

\noindent \textbf{Visual Studio} è un \acrog{ide} sviluppato da Microsoft utilizzato principalmente per lo sviluppo con linguaggi quali C\#, C++, F\#, Visual Basic e i \textit{framework} .NET e Xamarin.\\
Può essere configurato per lo sviluppo \textit{mobile} offrendo di conseguenza un ampio insieme di funzionalità specifiche, ad esempio l'esecuzione e il \textit{debug} su un simulatore, il \textit{deploy} e il \textit{debug} su dispositivo fisico, l'\textit{\glox{hotreload}} e la configurazione di diversi dispositivi su cui poter eseguire l'applicazione. Oltre a ciò, Visual Studio ha una funzione di correzione della sintassi e di suggerimento automatico.

\subsubsection{Microsoft SQL Server Management Studio}

\begin{figure}[H]
    \centering 
    \includegraphics[width=0.25\columnwidth]{images/loghi/sql_server.png} 
    \caption{Logo di Microsoft SQL Server Management Studio}
\end{figure}

\begin{itemize}
    \item \textbf{Versione}: 20.1.10.0;
    \item \textbf{Utilizzo}: gestione e interrogazione di \textit{database}.
\end{itemize}

\noindent \textbf{Microsoft SQL Server Management Studio (SSMS)} è un'applicazione Microsoft concepita per l'amministrazione di \textit{server} SQL Microsoft. Offre la possibilità di connettersi a \textit{server} SQL e visualizzarne i contenuti grazie alla funzionalità di esplorazione delle risorse. In questo modo è possibile visualizzare i \textit{database} presenti in un \textit{server} e tutte le relative tabelle, eseguire \textit{query} per visualizzare determinati dati oppure modificare la struttura o i contenuti delle tabelle.

\subsubsection{Swagger-UI}

\begin{figure}[H]
    \centering 
    \includegraphics[width=0.4\columnwidth]{images/loghi/swagger.png} 
    \caption{Logo di Swagger-UI}
\end{figure}

\begin{itemize}
    \item \textbf{Versione}: 3.1.0;
    \item \textbf{Utilizzo}: testing delle \acrog{api}.
\end{itemize}

\noindent \textbf{Swagger-UI} è uno strumento che offre un'interfaccia grafica per utilizzare le \acrog{api} all'interno di un progetto, senza necessariamente avere il codice dell'applicazione che le utilizza. In questo strumento, le \acrog{api} che richiedono parametri in \textit{input} forniscono già il \textit{template} dell'informazione in formato JSON, e, sempre nello stesso formato, restituiscono le risposte, che possono essere poi copiate o scaricate in un \textit{file} dedicato.\\
All'interno del progetto di \textit{stage} questo strumento è stato particolarmente utile durante la configurazione del progetto per l'esecuzione in locale e successivamente per verificare il corretto funzionamento delle nuove \acrog{api} sviluppate.


\subsection{Strumenti di analisi di rete}

\subsubsection{Wireshark}

\begin{figure}[H]
    \centering 
    \includegraphics[width=0.16\columnwidth]{images/loghi/wireshark.png} 
    \caption{Logo di Wireshark}
\end{figure}

\begin{itemize}
    \item \textbf{Versione}: 4.2.6;
    \item \textbf{Utilizzo}: analisi del traffico di rete in \textit{localhost}.
\end{itemize}

\noindent \textbf{Wireshark} è un \textit{software} \textit{open-source} e \textit{cross-platform} utilizzato per l'analisi della comunicazione all'interno di una rete e lo sviluppo di protocolli di comunicazione. Nella pagina principale permette la selezione della rete di cui si vuole analizzare il traffico, e, iniziata l'analisi, viene mostrata la lista di pacchetti di rete trasmessi, ognuno colorato in modo diverso secondo delle regole definite in automatico. Cliccando su un pacchetto è possibile poi vederne il contenuto.\\
All'interno del progetto, Wireshark è stato particolarmente utile nell'analizzare il traffico di rete in \textit{localhost} per verificare la sicurezza della trasmissione delle credenziali di autenticazione dell'app.


\subsection{Strumenti di collaborazione e gestione di progetto}

\subsubsection{Strumenti di Microsoft Office}

\begin{figure}[H]
    \begin{minipage}[b]{0.45\textwidth}
    \centering
    \includegraphics[width=.4\textwidth]{images/loghi/outlook-logo.png}
    \caption{Logo di Microsoft Outlook 2013}
    \end{minipage}
    \hfill
    \begin{minipage}[b]{0.45\textwidth}
    \centering
    \includegraphics[width=.4\textwidth]{images/loghi/Microsoft_Word.png}
    \caption{Logo di Microsoft Word 2013}
    \end{minipage}
\end{figure}

Due \textit{software} Microsoft molto utilizzati da VISIONEIMPRESA sono \textbf{Outlook} e \textbf{Word}. Il primo viene utilizzato per comunicare all'interno dell'azienda. Per comunicazioni più rapide e urgenti utilizzano l'applicazione \textbf{3CX}, ma per comunicazioni generali o per comunicare con gli stagisti viene utilizzato Outlook, all'interno del quale è stato creato un gruppo per facilitare l'invio di comunicazioni utili a tutti i dipendenti.\\
I documenti vengono invece scritti con Microsoft Word, e, se necessario, condivisi all'interno di un cloud aziendale chiamato Travaso, dove ogni dipendente ha una propria cartella personale. Tramite Word sono stati condivisi il piano di lavoro, il piano di test, l'analisi di moviEXPENSE e ho redatto la documentazione riguardante il lavoro svolto.


\subsubsection{Git}

\begin{figure}[H]
    \centering 
    \includegraphics[width=0.2\columnwidth]{images/loghi/Git.png} 
    \caption{Logo di Git}
\end{figure}

\noindent \textbf{Git} è un \textit{software} per il controllo di versione (VCS) ideato da Linus Torvalds. È un VCS distribuito, di conseguenza non esiste un \textit{database} centrale per la raccolta delle versioni, ma questo è distribuito ad ogni sviluppatore che sta lavorando al progetto.\\
Questa tipologia di \textit{software} permette di organizzare i flussi di lavoro in \textit{branch} (rami), facilitando quindi la collaborazione, e consente di raggruppare le modifiche effettuate al codice sorgente in \textit{commit} a cui viene associato un messaggio che descrive l'obiettivo di tali modifiche e un codice identificativo creato da Git; in questo modo è più semplice identificare le modifiche effettuate e, in caso sia necessario, riportare il \textit{software} ad uno stato ben preciso.\\
I VCS distribuiti sono spesso utilizzati in combinazione con un servizio di \textit{hosting} per il codice, ad esempio GitHub, GitLab o Bitbucket. Quest'ultimo è stato utilizzato durante lo \textit{stage} ed è descritto di seguito.


\subsubsection{\textit{Suite} Atlassian}

Oltre agli strumenti di Microsoft Office citati in precedenza, l'azienda fa un forte utilizzo della \textit{suite} \textbf{Atlassian}, in particolare di \textbf{Jira}, \textbf{Bitbucket} e \textbf{Confluence}. Durante lo \textit{stage} sono stati utilizzati questi ultimi due.

\vspace{6mm}

\begin{figure}[H]
    \centering 
    \includegraphics[width=0.4\columnwidth]{images/loghi/bitbucket.png} 
    \caption{Logo di Bitbucket}
\end{figure}

\paragraph{Bitbucket} Servizio di \textit{hosting} di \textit{repository} basato su Git. Permette di caricare i progetti a cui si sta lavorando contestualmente alla cronologia dei \textit{commit} di Git, condividendo così il lavoro svolto con tutti gli altri sviluppatori. Al suo interno è possibile creare \textit{branch} e visualizzare le modifiche effettuate in un preciso \textit{commit}.

\vspace{6mm}

\begin{figure}[H]
    \centering 
    \includegraphics[width=0.4\columnwidth]{images/loghi/confluence.png} 
    \caption{Logo di Confluence}
\end{figure}

\paragraph{Confluence} \textit{Software} per la collaborazione \textit{online} riguardo la stesura di documentazione. Viene utilizzato per scrivere la documentazione dei progetti \textit{software} a cui si sta lavorando, offrendo un \textit{editor} di testo integrato oppure permettendo di caricare documenti presenti in locale.


