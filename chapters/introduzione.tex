\chapter{Introduzione}
\label{cap:introduzione}

\section{Convenzioni tipografiche}

Durante la stesura del documento sono state adottate le seguenti convenzioni tipografiche:
\begin{enumerate}
    \item i termini tecnici, non di uso comune, ambigui, oltre che gli acronimi e le abbreviazioni, sono stati definiti nel glossario situato al termine del presente documento;
    \item la prima occorrenza di ogni termine presente nel glossario viene identificata nel seguente modo: \emph{termine di glossario}\glsfirstoccur;
    \item i termini in lingua inglese sono riportati in \textit{corsivo}.
\end{enumerate}

\section{L'azienda}

\begin{figure}[!h]
    \centering 
    \includegraphics[width=0.6\columnwidth]{images/logo-visioneimpresa.png} 
    \caption{Logo di VISIONEIMPRESA s.r.l. Società Benefit}
\end{figure}
\noindent VISIONEIMPRESA s.r.l. è una \textit{software house} nata negli anni Ottanta a Pernumia, in provincia di Padova. Dal 2016 fa parte di Office Group, un gruppo nato dall'unione di Office Information Technologies di Montegrotto Terme con altre aziende di Veneto e Lombardia.\\
L'obiettivo di VISIONEIMPRESA è lo sviluppo e la vendita di software gestionali per agevolare il lavoro delle piccole e medie aziende e contribuire alla loro digitalizzazione. I software sviluppati si dividono principalmente in due gruppi:
\begin{itemize}
    \item \textbf{software Vision}: il software principale è VisionEnterprise, ideato per migliorare la gestione dell'organizzazione aziendale nel suo complesso. In questo gruppo di software rientrano le soluzioni verticali, gestionali per specifiche tipologie di aziende, ad esempio VisionENERGY per il commercio di prodotti petroliferi o VisionFRESH per l'ingrosso di prodotti ortofrutticoli;
    \item \textbf{software movi}: applicazioni mobile specializzate in determinate azioni, ad esempio l'analisi dati (moviCHECK), l'invio di ordini (moviORDER) o l'invio di ticket per l'assistenza post-vendita (moviTICKET).
\end{itemize}
\noindent Nel 2023 VISIONEIMPRESA s.r.l. è diventata anche \gls{Società Benefit}\glsfirstoccur, impegnandosi così a perseguire obiettivi tra i quali rientrano il mantenimento di un buon equilibrio tra vita personale e lavoro dei dipendenti e la collaborazione con istituti di formazione del territorio (scuole superiori e università).

\section{Struttura del documento}

% TODO rivedere

\begin{description} 
    \item[{\hyperref[cap:descrizione-stage]{Il secondo capitolo}}] descrive il progetto di stage e lo scopo dell'applicazione.
    
    \item[{\hyperref[cap:tecnologie-strumenti]{Il terzo capitolo}}] presenta gli strumenti e le tecnologie utilizzate durante lo stage.
    
    \item[{\hyperref[cap:design]{Il quarto capitolo}}] descrive l'architettura e dell'applicazione.
    
    \item[{\hyperref[cap:codifica]{Il quinto capitolo}}] espone le principali modifiche e implementazioni effettuate durante lo stage.
    
    \item[{\hyperref[cap:verifica-validazione]{Nel sesto capitolo}}] viene spiegato come sono state effettuate le attività di verifica e validazione.
    
    \item[{\hyperref[cap:conclusioni]{Il settimo capitolo}}] espone le conclusioni tratte dall'esperienza di stage.
\end{description}
