\chapter{Introduzione}
\label{cap:introduzione}

\section{Convenzioni tipografiche}

\subsection{Acronimi, abbreviazioni, glossario}

Tutti gli acronimi, le abbreviazioni e i termini che necessitano di una definizione, poiché ambigui o non di uso comune, sono evidenziati in blu. La loro definizione è disponibile nelle apposite sezioni "Acronimi e abbreviazioni" e "Glossario" presenti al termine del presente documento. Se il documento è in consultazione digitalmente, è possibile cliccare sul termine desiderato per essere portati alla sua definizione.\\
I termini appartenenti al glossario si differenziano dagli altri dalla presenza di una "\textit{G}" al pedice.

\subsection{Elenchi}

Per migliorare la leggibilità, l'organizzazione e l'efficacia del testo, ho utilizzato degli elenchi puntati o numerati. L'uso di elenchi numerati è limitato ai casi in cui:
\begin{itemize}
    \item è presente una sequenza di azioni ordinate;
    \item è presente una gerarchia di priorità;
    \item alcuni punti dell'elenco devono essere citati in seguito.
\end{itemize}

\subsection{Stili di testo}

\subsubsection{\textbf{Grassetto}}

Utilizzato per evidenziare parole chiave all'interno del testo o evidenziare termini iniziali degli elenchi.

\subsubsection{\textit{Corsivo}}

Utilizzato per tutti i termini in lingua straniera o appartenenti al gergo tecnico.

\subsubsection{\texttt{Monospaziato}}

Utilizzato per i nomi di classi, metodi, tabelle dei \textit{database}, attributi, variabili, file.

\section{L'azienda}

\vspace{-3mm}

\begin{figure}[!h]
    \centering 
    \includegraphics[width=0.6\columnwidth]{images/logo-visioneimpresa.png} 
    \caption{Logo di VISIONEIMPRESA s.r.l. Società Benefit}
\end{figure}

\noindent VISIONEIMPRESA è una \textit{software house} nata negli anni Ottanta a Pernumia, in provincia di Padova. Dal 2016 fa parte di \textit{Office Group}, un gruppo nato dall'unione di \textit{Office Information Technologies} di Montegrotto Terme con altre aziende di Veneto e Lombardia.\\
L'obiettivo di VISIONEIMPRESA è lo sviluppo e la vendita di \textit{software} gestionali per agevolare il lavoro delle piccole e medie aziende e contribuire alla loro digitalizzazione. I \textit{software} sviluppati si dividono principalmente in due gruppi:
\begin{itemize}
    \item \textbf{\textit{software} Vision}: il \textit{software} primario è VisionEnterprise, ideato per migliorare la gestione dell'organizzazione aziendale nel suo complesso. In questo gruppo di \textit{software} rientrano le soluzioni verticali, gestionali per specifiche tipologie di aziende, ad esempio VisionENERGY per il commercio di prodotti petroliferi o VisionFRESH per l'ingrosso di prodotti ortofrutticoli;
    \item \textbf{\textit{software} movi}: applicazioni \textit{mobile} specializzate in determinate azioni, ad esempio l'analisi dati (moviCHECK), l'invio di ordini (moviORDER) o l'invio di ticket per l'assistenza post-vendita (moviTICKET).
\end{itemize}

\subsection{Società Benefit}

Nel 2023 VISIONEIMPRESA s.r.l. è diventata anche \textbf{società benefit}. Una società benefit non è altro che una tipologia di azienda che aggiunge a fianco degli obiettivi di profitto finanziario quello di avere un effetto positivo su ambiente e società. L'Italia è stato il primo Paese al di fuori degli USA ad introdurre questa tipologia di azienda.\\
VISIONEIMPRESA, quindi, si impegna a perseguire obiettivi tra i quali rientrano il mantenimento di un buon equilibrio tra vita personale e lavorativa dei dipendenti, la collaborazione con istituti di formazione del territorio (scuole superiori e università) e la riduzione dell'impatto ambientale dell'azienda.

\section{Struttura del documento}

\begin{description} 
    \item[{\hyperref[cap:descrizione-stage]{Il secondo capitolo}}] descrive il progetto di \textit{stage} e lo scopo dell'applicazione.
    
    \item[{\hyperref[cap:tecnologie-strumenti]{Il terzo capitolo}}] presenta gli strumenti e le tecnologie utilizzate durante lo \textit{stage}.
    
    \item[{\hyperref[cap:design]{Il quarto capitolo}}] descrive l'architettura e dell'applicazione.
    
    \item[{\hyperref[cap:codifica]{Il quinto capitolo}}] espone le principali modifiche effettuate e novità apportate, concludendo con una breve descrizione delle attività di verifica e validazione.
    
    \item[{\hyperref[cap:conclusioni]{Il sesto capitolo}}] espone le conclusioni tratte dall'esperienza di \textit{stage}.
\end{description}
