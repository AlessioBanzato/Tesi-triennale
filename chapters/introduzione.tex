\chapter{Introduzione}
\label{cap:introduzione}

\section{L'azienda}

\vspace{-3mm}

\begin{figure}[!h]
    \centering 
    \includegraphics[width=0.6\columnwidth]{images/logo-visioneimpresa.png} 
    \caption{Logo di VISIONEIMPRESA s.r.l. Società Benefit}
\end{figure}

\noindent VISIONEIMPRESA è una \textit{software house} nata negli anni Ottanta a Pernumia, in provincia di Padova. Dal 2016 fa parte di \textit{Office Group}, un gruppo nato dall'unione di \textit{Office Information Technologies} di Montegrotto Terme con altre aziende di Veneto e Lombardia.\\
L'obiettivo di VISIONEIMPRESA è lo sviluppo e la vendita di \textit{software} gestionali per agevolare il lavoro delle piccole e medie aziende e contribuire alla loro digitalizzazione. I \textit{software} sviluppati si dividono principalmente in due gruppi:
\begin{itemize}
    \item \textbf{\textit{software} Vision}: il \textit{software} primario è VisionEnterprise, ideato per migliorare la gestione dell'organizzazione aziendale nel suo complesso. In questo gruppo di \textit{software} rientrano le \textbf{soluzioni verticali}, gestionali per specifiche tipologie di aziende, ad esempio VisionENERGY per il commercio di prodotti petroliferi o VisionFRESH per l'ingrosso di prodotti ortofrutticoli;
    \item \textbf{\textit{software} movi}: applicazioni \textit{mobile} specializzate in determinate azioni, ad esempio l'analisi dati (moviCHECK), l'invio di ordini (moviORDER) o l'invio di \textit{ticket} per l'assistenza post-vendita (moviTICKET).
\end{itemize}

\subsection{Società Benefit}

Nel 2023 VISIONEIMPRESA s.r.l. è diventata anche \textbf{società benefit}. Una società benefit non è altro che una tipologia di azienda che aggiunge a fianco degli obiettivi di profitto finanziario quello di avere un effetto positivo su ambiente e società. L'Italia è stato il primo Paese al di fuori degli USA ad introdurre questa tipologia di azienda.\\
VISIONEIMPRESA, quindi, si impegna a perseguire obiettivi tra i quali rientrano il mantenimento di un buon equilibrio tra vita personale e lavorativa dei dipendenti, la collaborazione con istituti di formazione del territorio (scuole superiori e università) e la riduzione dell'impatto ambientale dell'azienda.


\section{Descrizione dello \textit{stage}}
\label{cap:descrizione-stage}

\subsection{Introduzione al progetto}

\begin{figure}[!h]
    \centering
    \includegraphics[width=0.5\columnwidth]{images/moviExpense_logo.png}
    \caption{Logo di moviEXPENSE}
\end{figure}

\noindent Il progetto di \textit{stage} si basava sullo studio e l'aggiornamento di moviEXPENSE, l'applicazione di VISIONEIMPRESA dedicata alla registrazione di note spese. L'app nasce con l'obiettivo di ottimizzare il processo di registrazione delle spese, cercando di ridurre a zero i rischi ad esso correlati.\\
La normale registrazione di spese avviene tramite accumulo di scontrini e fatture e compilazione di moduli. Tutti questi elementi vengono poi consegnati all'amministrazione dell'azienda che deciderà come gestirli. Questo processo, però, rischia di utilizzare diverse quantità di carta per la modulistica, e non è sicuro dal punto di vista della raccolta di fatture e scontrini, in quanto questi potrebbero essere persi per distrazione o potrebbero danneggiarsi in svariati modi.\\
Con moviEXPENSE, invece, non è più necessaria la modulistica cartacea poiché tutta trasposta nell'applicazione, e non c'è il rischio di perdita di scontrini e fatture in quanto possono essere salvati insieme alla spesa tramite foto.\\
Un altro problema che moviEXPENSE risolve è quello della comunicazione con l'amministrazione aziendale: dal momento in cui una spesa viene salvata nell'applicazione, è subito visibile dall'amministrazione, in quanto i dati sono tutti salvati su \textit{cloud}, e di conseguenza è immediatamente gestibile, andando così a ridurre drasticamente i tempi morti del processo.

\subsection{Obiettivi}
\label{cap:obiettivi}
\subsubsection{Obbligatori}

\begin{itemize}
    \item Studio degli strumenti e delle tecnologie da utilizzare, esposte nel dettaglio nel capitolo successivo;
    \item Analisi e revisione del sistema di autenticazione dell'app, in particolare:
    \begin{itemize}
        \item comprensione dei meccanismi di autenticazione presenti;
        \item correzione delle \acrog{api} di autenticazione;
        \item correzione delle \acrog{api} di collegamento ai \textit{database}.
    \end{itemize}
    \item Implementazione di ulteriori controlli per rendere più sicura e veloce la registrazione delle spese;
    \item Correzione degli errori presenti nell'applicazione, segnalati nell'analisi condivisa a inizio \textit{stage};
    \item Implementazione delle novità introdotte nell'analisi condivisa a inizio \textit{stage};
    \item Rivisitazione dell'interfaccia grafica su \textit{smartphone};
    \item \textit{Testing} dell'applicazione.
\end{itemize}

\subsubsection{Desiderabili}

\begin{itemize}
    \item Creazione della documentazione con i seguenti livelli di priorità decrescente:
    \begin{enumerate}
        \item Documentazione tecnica delle modifiche effettuate e delle novità introdotte;
        \item Manuale utente delle modifiche effettuate e delle novità introdotte;
        \item Documentazione tecnica generale dell'applicazione;
        \item Manuale utente generale dell'applicazione.
    \end{enumerate}
\end{itemize}

\subsubsection{Opzionali}

\begin{itemize}
    \item Rivisitazione dell'interfaccia grafica per uso su \textit{tablet}.
\end{itemize}


\subsection{Pianificazione e svolgimento del lavoro}

\subsubsection{Pianificazione}

Le attività da svolgere durante il periodo di \textit{stage} sono state distribuite nell'arco di otto settimane, per un totale di 300 ore di lavoro. Di seguito è riportata nel dettaglio la suddivisione delle attività per periodi:
\begin{enumerate}
    \item \textbf{Studio}: settimana dedicata allo studio dell'applicazione e delle tecnologie da utilizzare;
    \item \textbf{Autenticazione}: miglioramento del precedente sistema di autenticazione, da effettuare basandosi sul lavoro svolto su altre app, in particolare moviORDER e moviCHECK;
    \item \textbf{\textit{Bug fixing}}: correzione degli errori presenti nell'applicazione;
    \item \textbf{Novità}: implementazione di nuove funzionalità e ampliamento di quelle già esistenti;
    \item \textbf{Interfacce}: rinnovamento dell'interfaccia grafica ispirato dai \textit{\glox{mockup}} di moviORDER;
    \item \textbf{Test e documentazione}: \textit{testing} dell'applicazione e redazione della documentazione richiesta.
\end{enumerate}
Nella seguente tabella sono riportati i periodi sopra citati e la loro suddivisione settimanale e oraria.

\renewcommand{\arraystretch}{1.3}
\begin{table}[H]
    \centering
        \begin{tabular}{| c | c | c |}
        \hline
        \textbf{Settimana} & \textbf{Attività} & \textbf{Ore} \\
        \hline
        1 & Studio & 40 \\
        \hline
        2 & Autenticazione & 40 \\
        \hline
        3 & \textit{Bug fixing} & 40 \\
        \hline
        4-5 & Novità & 80 \\
        \hline
        6 & Interfacce & 40 \\
        \hline
        7-8 & Test e documentazione & 60 \\
        \hline
        \multicolumn{3}{c}{\rule{0pt}{1em}} \\
        \hline
        \multicolumn{2}{|c|}{\textbf{Totale}} & 300 \\
        \hline
        \end{tabular}
        \caption{Pianificazione del lavoro per settimane e ore}
\end{table}
\renewcommand{\arraystretch}{1}

\subsubsection{Svolgimento}

Lo \textit{stage} si è svolto completamente in presenza a Pernumia, presso la sede di VISIONEIMPRESA. All'inizio dei lavori sono stato accolto dall'amministratore, il quale mi ha presentato l'azienda, i principali \textit{software} da loro sviluppati e ha fatto una panoramica sul mio progetto.\\
Successivamente mi ha condiviso un documento di analisi redatto da lui e da un utilizzatore di moviEXPENSE presso \textit{Office Group}. In questo documento erano presenti tutti gli aspetti critici dell'applicazione e tutte le novità che dovevano essere apportate per migliorare l'app e il suo utilizzo.\\
Tutto il lavoro è stato svolto utilizzando un \textit{computer} aziendale con sistema operativo Windows 10, e uno \textit{smartphone} aziendale nel quale eseguire l'applicazione.\\
Durante lo \textit{stage} ho lavorato in modo indipendente, avendo come riferimenti principali:
\begin{itemize}
    \item \textbf{Francesco Turra}: amministratore dell'azienda, consulente, analista, con il quale sono stato in contatto per esporre l'andamento del lavoro e validare le modifiche e implementazioni;
    \item \textbf{Paolo Stefani} (tutor aziendale) e altri programmatori: per qualsiasi problema o difficoltà con il codice o il \textit{setup} dell'ambiente di sviluppo. Con Paolo in particolare anche per la validazione finale del lavoro svolto.
\end{itemize}



\section{Struttura del documento}

\begin{description}
    \item[{\hyperref[cap:tecnologie-strumenti]{Il secondo capitolo}}] presenta gli strumenti e le tecnologie utilizzate durante lo \textit{stage}.
    
    \item[{\hyperref[cap:design]{Il terzo capitolo}}] descrive l'architettura dell'applicazione.
    
    \item[{\hyperref[cap:codifica]{Il quarto capitolo}}] espone le principali modifiche effettuate e novità apportate, concludendo con una breve descrizione delle attività di verifica e validazione.
    
    \item[{\hyperref[cap:conclusioni]{Il quinto capitolo}}] espone le conclusioni tratte dall'esperienza di \textit{stage}.
\end{description}

\section{Convenzioni tipografiche}

\subsection{Acronimi, abbreviazioni, glossario}

Tutti gli acronimi, le abbreviazioni e i termini che necessitano di una definizione, poiché ambigui o non di uso comune, sono evidenziati in blu. La loro definizione è disponibile nelle apposite sezioni "Acronimi e abbreviazioni" e "Glossario" presenti al termine del documento. Se il documento è in consultazione digitalmente, è possibile cliccare sul termine desiderato per essere portati alla sua definizione.\\
I termini appartenenti al glossario si differenziano dagli altri dalla presenza di una "\textit{G}" al pedice.

\subsection{Elenchi}

Per migliorare la leggibilità, l'organizzazione e l'efficacia del testo, ho utilizzato degli elenchi puntati o numerati. L'uso di elenchi numerati è limitato ai casi in cui:
\begin{itemize}
    \item è presente una sequenza di azioni ordinate;
    \item è presente una gerarchia di priorità;
    \item alcuni punti dell'elenco devono essere citati in seguito.
\end{itemize}

\subsection{Stili di testo}

\subsubsection{\textbf{Grassetto}}

Utilizzato per evidenziare parole chiave all'interno del testo o evidenziare termini iniziali degli elenchi.

\subsubsection{\textit{Corsivo}}

Utilizzato per tutti i termini in lingua straniera o appartenenti al gergo tecnico.

\subsubsection{\texttt{Monospaziato}}

Utilizzato per i nomi di classi, metodi, tabelle dei \textit{database}, attributi, variabili, file.
