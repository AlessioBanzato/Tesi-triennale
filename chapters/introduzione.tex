\chapter{Introduzione}
\label{cap:introduzione}

Questo capitolo presenta:
\begin{itemize}
    \item la dichiarazione delle convenzioni tipografiche adottate;
    \item una breve descrizione dell'azienda presso cui è stato svolto lo stage;
    \item una breve illustrazione della struttura del presente documento.
\end{itemize}

\section{Convenzioni tipografiche}

Durante la stesura del documento sono state adottate le seguenti convenzioni tipografiche:
\begin{enumerate}
    \item i termini tecnici, non si uso comune, ambigui, oltre che gli acronimi e le abbreviazioni, sono stati definiti nel glossario situato al termine del presente documento;
    \item la prima occorrenza di ogni termine presente nel glossario viene identificata nel seguente modo; \emph{termine di glossario}\glsfirstoccur;
    \item i termini in lingua inglese sono riportati in \textit{corsivo}.
\end{enumerate}

\section{L'azienda}

\begin{figure}[!h]
    \centering 
    \includegraphics[width=0.7\columnwidth]{images/logo-visioneimpresa.png} 
    \caption{Logo di VISIONEIMPRESA s.r.l. Società Benefit}
\end{figure}

VISIONEIMPRESA s.r.l. è una \textit{software house} nata negli anni Ottanta nella provincia di Padova, in particolare a Pernumia, vicino alla zona dei colli. Dal 2016 fa parte di Office Group, un gruppo nato dall'unione di Office Information Technologies di Montegrotto Terme con altre aziende di Veneto e Lombardia.\\
L'obiettivo di VISIONEIMPRESA è lo sviluppo e la vendita di software gestionali per agevolare il lavoro delle aziende e contribuire alla digitalizzazione di esse. I software sviluppati si dividono principalmente in due gruppi:
\begin{itemize}
    \item \textbf{software Vision}: il software principale è VisionEnterprise, ideato per migliorare la gestione dell'organizzazione aziendale nel suo complesso. In questo gruppo di software rientrano le soluzioni verticali, gestionali per specifiche tipologie di aziende, ad esempio VisionENERGY per il commercio di prodotti petroliferi o VisionFRESH per l'ingrosso di prodotti ortofrutticoli;
    \item \textbf{software movi}: applicazioni mobile specializzate in determinate azioni, ad esempio l'analisi dati (moviCHECK), l'invio di ordini (moviORDER) o l'inivio di ticket per l'assistenza post-vendita (moviTICKET).
\end{itemize}

% Inserire discorso su Società Benefit

\section{Struttura del documento}

\begin{description} 
    \item[{\hyperref[cap:descrizione-stage]{Il secondo capitolo}}] descrive il progetto di stage e lo scopo dell'applicazione.
    
    \item[{\hyperref[cap:strumenti-tecnologie]{Il terzo capitolo}}] presenta gli strumenti e le tecnologie utilizzate durante lo stage.
    
    \item[{\hyperref[cap:analisi-requisiti]{Il quarto capitolo}}] espone le funzionalità dell'applicazione nella forma di requisiti e casi d'uso.
    
    \item[{\hyperref[cap:progettazione-codifica]{Il quinto capitolo}}] descrive l'architettura e il design dell'applicazione e come queste sono state codificate.
    
    \item[{\hyperref[cap:verifica-validazione]{Nel sesto capitolo}}] vengono spiegati i test effettuati.
    
    \item[{\hyperref[cap:conclusioni]{Il settimo capitolo}}] espone le conclusioni tratte dall'esperienza di stage.
\end{description}
