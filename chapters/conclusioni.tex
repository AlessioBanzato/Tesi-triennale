\chapter{Conclusioni}
\label{cap:conclusioni}

\intro{In questo capitolo vengono presentate le conclusioni tratte da questa esperienza di \textit{stage}. La prima parte del capitolo si concentra sull'organizzazione e gli obiettivi del progetto, mentre la seconda parte è incentrata sulle riflessioni svolte a termine dell'esperienza.}

\section{Consuntivo orario}

\renewcommand{\arraystretch}{1.3}
\begin{table}[H]
    \centering
        \begin{tabular}{| c | c | c | c |} 
        \hline
        \textbf{Settimana} & \textbf{Attività} & \textbf{Ore} & \textbf{Variazione}\\
        \hline
        1 & Studio & 50 & +10 \\ 
        \hline
        2 & Autenticazione & 40 & - \\
        \hline
        3-4-5 & \emph{Bug fixing} e novità & 120 & - \\
        \hline
        6 & Interfacce & 40 & - \\
        \hline
        7-8 & Test e documentazione & 50 & -10\\
        \hline
        \multicolumn{4}{c}{\rule{0pt}{1em}} \\
        \hline
        \multicolumn{2}{|c|}{\textbf{Totale}} & 300 & $\pm$0\\
        \hline
        \end{tabular}
        \caption{Consuntivo del lavoro per settimane e ore}
\end{table}
\renewcommand{\arraystretch}{1}

\noindent Come si può notare dalla tabella, ci sono state delle piccole variazioni nel numero di ore di alcune attività, ma queste non sono andate ad influenzare il monte ore finale.\\
Per quanto riguarda, invece, la distribuzione di tali attività nelle settimane, "\emph{bug fixing}" e "novità" sono state inserite nella stessa riga, occupando le settimane dalla 3 alla 5. Questo perché le due attività sono state svolte insieme, e non in modo sequenziale, in quanto l'implementazione delle richieste sull'applicazione ha seguito l'ordine presente nel documento in cui tali richieste erano esposte, nel quale non era presente una divisione netta tra le richieste di \emph{bug fixing} e di introduzione di novità.\\
La prima settimana di lavoro è durata più di quanto pianificato perché la configurazione dell'applicazione per la sua esecuzione in locale ha richiesto più tempo del previsto.

\section{Obiettivi raggiunti}

Al termine dello \textit{stage} ho raggiunto tutti gli obiettivi obbligatori esposti nella \sezref{cap:obiettivi} e quasi tutti i desiderabili. Le uniche due attività non svolte sono state la creazione del manuale utente dell'applicazione e la rivisitazione dell'interfaccia grafica per uso su \emph{tablet}, a causa di mancanza di tempo, ma dato che avevano entrambe una bassa priorità, il livello di soddisfazione da parte dell'azienda è stato comunque alto.\\
Di seguito è riportata una tabella con le percentuali di raggiungimento degli obiettivi.

\renewcommand{\arraystretch}{1.3}
\begin{table}[H]
    \centering
        \begin{tabular}{| c | c |} 
        \hline
        \textbf{Categoria obiettivi} & \textbf{Percentuale di raggiungimento}\\
        \hline
        Obbligatori & 100\% \\
        \hline
        Desiderabili & 75\% \\
        \hline
        Opzionali & 0\% \\
        \hline
        \multicolumn{2}{c}{\rule{0pt}{1em}} \\
        \hline
        \textbf{Totale} & 83,33\%\tablefootnote{Ho utilizzato il numero di obiettivi di ogni categoria come "peso" per il calcolo della percentuale totale.$$P_{tot}=\frac{n_1P_1+n_2P_2+n_3P_3}{n_1+n_2+n_3}=\frac{7\cdot100+4\cdot75+1\cdot0}{7+4+1}=83,33$$} \\
        \hline
        \end{tabular}
        \caption{Percentuali di raggiungimento obiettivi per categorie}
\end{table}
\renewcommand{\arraystretch}{1}


\section{Conoscenze acquisite}

Grazie a questo \textit{stage} ho potuto affacciarmi al mondo della programmazione \textit{mobile}, argomento che non avevo trattato né all'università e nemmeno individualmente, poiché non mi attirava in modo particolare. In questo contesto ho migliorato le mie capacità di autoapprendimento legate sia allo studio di nuovi strumenti e tecnologie che all'utilizzo e lo studio di software realizzato da altri, comprendendone l'architettura, l'organizzazione delle cartelle e lo stile di codifica.\\
Benché il \textit{framework} utilizzato per il \textit{frontend} (Xamarin) non sia più supportato, ho comunque ritenuto utile il suo studio, in quanto è in ogni caso la base di partenza per il \textit{framework} che l'ha sostituito, ovvero .NET MAUI, e perciò se in futuro mi verrà chiesto di sviluppare applicazioni utilizzando questo nuovo \textit{framework}, avrò già una conoscenza di base che mi permetterà di apprenderlo più facilmente.


\section{Valutazione personale}

Nel complesso ho ritenuto lo \textit{stage} un'esperienza molto utile, in quanto mi ha permesso innanzitutto di interfacciarmi con il mondo del lavoro, in particolare con la "vita d'ufficio", esperienza che non avevo mai svolto. Ritengo quindi importante anche lo svolgimento di un'esperienza di questo tipo in presenza piuttosto che da remoto, specialmente se è la prima esperienza lavorativa che si affronta, in quanto la comunicazione con i colleghi risulta più rapida ed efficiente rispetto ad una comunicazione via mail, messaggio o chiamata, nelle quali possono presentarsi diversi problemi, dalla mancata visualizzazione di un messaggio a problemi audio/video.\\
Lo \textit{stage} mi ha anche fatto ricredere riguardo la programmazione \textit{mobile}, la quale è risultata molto affascinante, anche durante l'attività di \emph{restyling}, che non avrei pensato mi sarebbe interessata particolarmente.\\
L'ambiente che ho trovato presso VISIONEIMPRESA, inoltre, è stato un ambiente molto accogliente, amichevole e con persone fortemente disponibili in caso avessi dubbi o difficoltà, e ciò ha contribuito a rendere piacevole questa esperienza, evitando di essere vissuta come semplice attività obbligatoria del piano di studi.\\
Infine ritengo che effettuare uno \textit{stage} al termine del percorso di studi sia particolarmente utile, poiché permette di:
\begin{itemize}
    \item avvicinarsi ad argomenti poco conosciuti e quindi apprendere strumenti, tecnologie e metodologie nuove che possono arricchire il proprio bagaglio culturale e professionale;
    \item confermare o meno interessi per futuri lavori o percorsi di studio;
    \item conoscere persone nell'ambito professionale di interesse, con cui scambiare opinioni e conoscere meglio l'ambiente lavorativo.
\end{itemize}
