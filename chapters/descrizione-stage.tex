\chapter{Descrizione dello stage}
\label{cap:descrizione-stage}

\intro{Questo capitolo presenta una panoramica sul progetto di stage, esponendo in particolare:
\begin{itemize}
    \item un'introduzione al progetto;
    \item i suoi obiettivi;
    \item la pianificazione del lavoro.
\end{itemize}
}

\section{Introduzione al progetto}

\begin{figure}[!h]
    \centering 
    \includegraphics[width=0.5\columnwidth]{images/moviExpense_logo.png} 
    \caption{Logo di moviEXPENSE}
\end{figure}

Il progetto di stage si basava sullo studio e l'aggiornamento di moviEXPENSE, l'applicazione di VISIONEIMPRESA dedicata alla registrazione di note spese. L'app nasce con l'obiettivo di ottimizzare il processo di registrazione delle spese, cercando di ridurre a zero i rischi ad esso intrinseci.\\
La normale registrazione di spese avviene tramite accumulo di scontrini e fatture e compilazione di moduli. Tutti questi elementi verranno poi consegnati all'amministrazione dell'azienda che deciderà come gestirli. Questo processo, però, rischia di utilizzare diverse quantità di carta per la modulistica, e non è sicuro dal punto di vista della raccolta di fatture e scontrini, in quanto questi potrebbero essere persi per distrazione o potrebbero danneggiarsi in svariati modi.\\
Con moviEXPENSE, invece, non è più necessaria la modulistica cartacea poiché tutta trasposta nell'applicazione, e non c'è il rischio di perdita di scontrini e fatture in quanto possono essere salvati insieme alla nota tramite foto.\\
Un altro problema che moviEXPENSE risolve è quello della comunicazione con l'amministrazione aziendale: dal momento in cui una spesa viene salvata nell'applicazione, è subito visibile dall'amministrazione, in quanto i dati sono tutti salvati su \emph{\glox{cloud}}, e di conseguenza è immediatamente gestibile, andando così a ridurre drasticamente i tempi morti del processo.

\section{Obiettivi}
\label{cap:obiettivi}
\subsection{Obbligatori}

\begin{itemize}
    \item Studio degli strumenti e delle tecnologie da utilizzare, esposte nel dettaglio nel capitolo successivo;
    \item Analisi e revisione del sistema di autenticazione dell'app:
    \begin{itemize}
        \item Comprensione dei meccanismi di autenticazione presenti;
        \item Correzione delle \glox{API} di autenticazione;
        \item Correzione delle \glox{API} di collegamento ai database.
    \end{itemize}
    \item Implementazione di ulteriori controlli per rendere più sicura e veloce la registrazione delle spese;
    \item Correzione degli errori presenti nell'applicazione, segnalati nell'analisi condivisa a inizio stage;
    \item Implementazione delle novità introdotte nell'analisi condivisa a inizio stage;
    \item Rivisitazione dell'interfaccia grafica su \textit{smartphone};
    \item \textit{Testing} dell'applicazione.
\end{itemize}

\subsection{Desiderabili}

\begin{itemize}
    \item Creazione della documentazione con i seguenti livelli di priorità decrescente:
    \begin{enumerate}
        \item Documentazione tecnica delle modifiche effettuate e delle novità introdotte;
        \item Manuale utente delle modifiche effettuate e delle novità introdotte;
        \item Documentazione tecnica generale dell'applicazione;
        \item Manuale utente generale dell'applicazione.
    \end{enumerate}
\end{itemize}

\subsection{Opzionali}

\begin{itemize}
    \item Rivisitazione dell'interfaccia grafica per uso su \textit{tablet}.
\end{itemize}


\section{Pianificazione del lavoro}

Le attività da svolgere durante il periodo di stage sono state distribuite nell'arco di otto settimane, per un totale di 300 ore di lavoro. Di seguito è riportata nel dettaglio la suddivisione delle attività per periodi:
\begin{enumerate}
    \item \textbf{Studio}: settimana dedicata allo studio dell'applicazione e delle tecnologie da utilizzare;
    \item \textbf{Autenticazione}: miglioramento del precedente sistema di autenticazione, effettuato basandosi sul lavoro svolto su altre app, in particolare moviORDER e moviCHECK;
    \item \textbf{\textit{Bug fixing}}: correzione degli errori presenti nell'applicazione;
    \item \textbf{Novità}: implementazione di nuove funzionalità e ampliamento di quelle già esistenti;
    \item \textbf{Interfacce}: rinnovamento dell'interfaccia grafica ispirato dai mockup di moviORDER;
    \item \textbf{Test e documentazione}: \emph{testing} e redazione della documentazione richiesta.
\end{enumerate}
Nella seguente tabella sono riportati i periodi sopra citati e la loro suddivisione settimanale e oraria.

\renewcommand{\arraystretch}{1.3}
\begin{table}[H]
    \centering
        \begin{tabular}{| c | c | c |} 
        \hline
        \textbf{Settimana} & \textbf{Attività} & \textbf{Ore} \\
        \hline
        1 & Studio & 40 \\ 
        \hline
        2 & Autenticazione & 40 \\
        \hline
        3 & \emph{Bug fixing} & 40 \\
        \hline
        4-5 & Novità & 80 \\
        \hline
        6 & Interfacce & 40 \\
        \hline
        7-8 & Test e documentazione & 60 \\
        \hline
        \multicolumn{3}{c}{\rule{0pt}{1em}} \\
        \hline
        \multicolumn{2}{|c|}{\textbf{Totale}} & 300 \\
        \hline
        \end{tabular}
        \caption{Pianificazione del lavoro per settimane e ore}
\end{table}
\renewcommand{\arraystretch}{1}