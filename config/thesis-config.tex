% \omiss produces '[...]'
\newcommand{\omissis}{[\dots\negthinspace]}

% Itemize symbols
% see: https://tex.stackexchange.com/a/62497
% \renewcommand{\labelitemi}{$\bullet$}
% \renewcommand{\labelitemii}{$\cdot$}
% \renewcommand{\labelitemiii}{$\diamond$}
% \renewcommand{\labelitemiv}{$\ast$}


\let\Chaptermark\chaptermark
% \Chaptername gives current chapter name
\def\chaptermark#1{\def\Chaptername{#1}\Chaptermark{#1}}
\makeatletter
% \currentname gives the current section name
\newcommand*{\currentname}{\@currentlabelname}
\makeatother

% Uncomment the following line for a different header/footer style
\pagestyle{fancy} \setlength{\headheight}{14.5pt}
\fancyhead[L]{\nouppercase{\slshape \leftmark}}
\fancyhead[R]{}
% Page number always in footer
\cfoot{\thepage}


% Custom hyphenation rules
\hyphenation {
    e-sem-pio
    ex-am-ple
}

% Images path, not using \graphicspath because it doesn't properly work with
% latexmk custom dependencies
\NewCommandCopy{\latexincludegraphics}{\includegraphics}
%\renewcommand{\includegraphics}[2][]{\latexincludegraphics[#1]{../images/#2}}

% Page format settings
% see: http://wwwcdf.pd.infn.it/AppuntiLinux/a2547.htm
\setlength{\parindent}{14pt}    % first row indentation
\setlength{\parskip}{0pt}       % paragraphs spacing


% Load variables
\newcommand{\myName}{Alessio Banzato}
\newcommand{\myID}{2042381}
\newcommand{\myTitle}{moviEXPENSE 2: studio e aggiornamento di un'applicazione mobile per la registrazione di note spese}
\newcommand{\myDegree}{Tesi di laurea}
\newcommand{\myUni}{Università degli Studi di Padova}
\newcommand{\myFaculty}{Corso di Laurea in Informatica}
\newcommand{\myDepartment}{Dipartimento di Matematica ``Tullio Levi-Civita''}
\newcommand{\profTitle}{Prof.}
\newcommand{\myProf}{Francesco Ranzato}
\newcommand{\myLocation}{Padova}
\newcommand{\myAA}{2023-2024}
\newcommand{\myTime}{Settembre 2024}

% PDF file metadata fields
% when updating them delete the build directory, otherwise they won't change
\begin{filecontents*}{\jobname.xmpdata}
  \Title{moviEXPENSE 2: studio e aggiornamento di un'applicazione mobile per la registrazione di note spese}
  \Author{Alessio Banzato}
  \Language{it-IT}
  \Subject{}
  \Keywords{mobile\sep applicazione\sep gestionale}
\end{filecontents*}


% Acronyms
\newacronym[description={\glslink{API}{Application Program Interface}}]
    {api}{API}{Application Program Interface}

\newacronym[description={\glslink{IDE}{Integrated Development Environment}}]
    {ide}{IDE}{Integrated Development Environment}

\newacronym[description={\glslink{JWT}{JSON Web Token}}]
    {jwt}{JWT}{JSON Web Token}

\newacronym{ssms}{SSMS}{SQL Server Management Studio}

\newacronym{ui}{UI}{User Interface (Interfaccia Utente)}

% Glossary entries
\newglossaryentry{API} {
    name={Application Program Interface},
    text=API,
    sort=api,
    description={In italiano, interfaccia di programmazione di un'applicazione. Insieme di regole e di protocolli che permette la comunicazione e lo scambio di dati tra più \textit{software} o parti di \textit{software}}
}

\newglossaryentry{code-behind}{
    name=Code-behind,
    text=code-behind,
    sort=code-behind,
    description={Pratica di porre in due \textit{file} separati il codice che gestisce la struttura e presentazione di un'applicazione e quello che ne gestisce il comportamento. È una pratica utilizzata principalmente nella programmazione in ambienti Microsoft dove il codice di struttura e presentazione è definito nei \textit{file} \texttt{.xaml}, mentre il comportamento è definito nei \textit{file} \texttt{.cs}}
}

\newglossaryentry{mockup}{
    name=Mockup,
    text=mockup,
    sort=mockup,
    description={Modello utilizzato per mostrare il \textit{design} di un prodotto. In informatica è il risultato della progettazione di un'interfaccia grafica, utilizzato come riferimento per la sua implementazione}
}

\newglossaryentry{codebase}{
    name=Codebase,
    text=codebase,
    sort=codebase,
    description={Insieme del codice sorgente che sta alla base di un \textit{software}}
}

\newglossaryentry{libreria}{
    name=Libreria,
    text=librerie,
    sort=libreria,
    description={Insieme di risorse in sola lettura che possono essere incluse e utilizzate all'interno di un \textit{software}}
}

\newglossaryentry{IDE}{
    name={Integrated Development Environment},
    text=IDE,
    sort=ide,
    description={In italiano, ambiente di sviluppo integrato. Applicazione utilizzata per lo sviluppo \textit{software} che offre, oltre ad un \textit{editor} di testo, diverse funzionalità che aiutano il programmatore durante la scrittura del codice, ad esempio sistemi di automazione e \textit{debugging}}
}

\newglossaryentry{hotreload}{
    name={Hot reload},
    text={hot reload},
    sort=hotreload,
    description={Funzionalità presente in alcuni \acrog{ide} che permette di applicare le modifiche effettuate ad un'applicazione durante la sua esecuzione, senza dover quindi chiudere l'app e rieseguirla}
}

\newglossaryentry{JWT}{
    name={JSON Web Token},
    text=JWT,
    sort=jwt,
    description={Standard \textit{web} per lo scambio di dati basato su un oggetto chiamato \textit{token} suddiviso in tre parti: \textit{header}, che contiene le informazioni che identificano l'algoritmo di codifica utilizzato, \textit{payload}, che contiene i dati codificati e in formato JSON, e una \textit{signature}, ovvero una firma che dà validità al \textit{token}}
  }

  \newglossaryentry{garbagecollector}{
    name={Garbage collector},
    text={garbage collector},
    sort=garbagecollector,
    description={Sistema di gestione automatica della memoria allocata da un programma in esecuzione. Rileva i blocchi di memoria non più utilizzati e li libera per migliorare le \textit{performance} del programma che sta eseguendo}
  }

\makeglossaries

\bibliography{appendix/bibliography}

\defbibheading{bibliography} {
    \cleardoublepage
    \phantomsection
    \addcontentsline{toc}{chapter}{\bibname}
    \chapter*{\bibname\markboth{\bibname}{\bibname}}
}

% Spacing between entries
\setlength\bibitemsep{1.5\itemsep}

\DeclareBibliographyCategory{opere}
\DeclareBibliographyCategory{web}

\addtocategory{opere}{womak:lean-thinking}
\addtocategory{web}{site:agile-manifesto}

\defbibheading{opere}{\section*{Riferimenti bibliografici}}
\defbibheading{web}{\section*{Siti Web consultati}}


\captionsetup{
    tableposition=bottom,
    figureposition=top,
    font=small,
    format=hang,
    labelfont=bf
}

\hypersetup{
    %hyperfootnotes=false,
    %pdfpagelabels,
    colorlinks=true,
    linktocpage=true,
    pdfstartpage=1,
    pdfstartview=,
    breaklinks=true,
    pdfpagemode=UseNone,
    pageanchor=true,
    pdfpagemode=UseOutlines,
    plainpages=false,
    bookmarksnumbered,
    bookmarksopen=true,
    bookmarksopenlevel=1,
    hypertexnames=true,
    pdfhighlight=/O,
    %nesting=true,
    %frenchlinks,
    urlcolor=webbrown,
    linkcolor=RoyalBlue,
    citecolor=webgreen
    %pagecolor=RoyalBlue,
}

% Delete all links and show them in black
\if \isprintable 1
    \hypersetup{draft}
\fi

% Listings setup
\lstset{
    language=[LaTeX]Tex,%C++,
    keywordstyle=\color{RoyalBlue}, %\bfseries,
    basicstyle=\small\ttfamily,
    %identifierstyle=\color{NavyBlue},
    commentstyle=\color{Green}\ttfamily,
    stringstyle=\rmfamily,
    numbers=none, %left,%
    numberstyle=\scriptsize, %\tiny
    stepnumber=5,
    numbersep=8pt,
    showstringspaces=false,
    breaklines=true,
    frameround=ftff,
    frame=single
}

\definecolor{webgreen}{rgb}{0,.5,0}
\definecolor{webbrown}{rgb}{.6,0,0}

\newcommand{\sectionname}{sezione}
\addto\captionsitalian{\renewcommand{\figurename}{Figura}
                       \renewcommand{\tablename}{Tabella}}

\newcommand{\glsfirstoccur}{\ap{{[g]}}}

\newcommand{\intro}[1]{\emph{\textsf{#1}}}

% Risks environment
\newcounter{riskcounter}                % define a counter
\setcounter{riskcounter}{0}             % set the counter to some initial value

%%%% Parameters
% #1: Title
\newenvironment{risk}[1]{
    \refstepcounter{riskcounter}        % increment counter
    \par \noindent                      % start new paragraph
    \textbf{\arabic{riskcounter}. #1}   % display the title before the content of the environment is displayed
}{
    \par\medskip
}

\newcommand{\riskname}{Rischio}

\newcommand{\riskdescription}[1]{\textbf{\\Descrizione:} #1.}

\newcommand{\risksolution}[1]{\textbf{\\Soluzione:} #1.}

% Use case environment
\newcounter{usecasecounter}             % define a counter
\setcounter{usecasecounter}{0}          % set the counter to some initial value

%%%% Parameters
% #1: ID
% #2: Nome
\newenvironment{usecase}[2]{
    \renewcommand{\theusecasecounter}{\usecasename #1}  % this is where the display of
                                                        % the counter is overwritten/modified
    \refstepcounter{usecasecounter}             % increment counter
    \vspace{10pt}
    \par \noindent                              % start new paragraph
    {\large \textbf{\usecasename #1: #2}}       % display the title before the
                                                % content of the environment is displayed
    \medskip
}{
    \medskip
}

\newcommand{\usecasename}{UC}

\newcommand{\usecaseactors}[1]{\textbf{\\Attori Principali:} #1. \vspace{4pt}}
\newcommand{\usecasepre}[1]{\textbf{\\Precondizioni:} #1. \vspace{4pt}}
\newcommand{\usecasedesc}[1]{\textbf{\\Descrizione:} #1. \vspace{4pt}}
\newcommand{\usecasepost}[1]{\textbf{\\Postcondizioni:} #1. \vspace{4pt}}
\newcommand{\usecasealt}[1]{\textbf{\\Scenario Alternativo:} #1. \vspace{4pt}}

% Namespace description environment
\newenvironment{namespacedesc}{
    \vspace{10pt}
    \par \noindent  % start new paragraph
    \begin{description}
}{
    \end{description}
    \medskip
}

\newcommand{\classdesc}[2]{\item[\textbf{#1:}] #2}

\newcommand{\figref}[1]{Figura~\ref{#1}}
\newcommand{\sezref}[1]{Sezione~\ref{#1}}
