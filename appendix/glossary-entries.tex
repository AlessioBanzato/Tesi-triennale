% Acronyms
\newacronym[description={\glslink{API}{Application Program Interface}}]
    {api}{API}{Application Program Interface}

\newacronym[description={\glslink{IDE}{Integrated Development Environment}}]
    {ide}{IDE}{Integrated Development Environment}

\newacronym[description={\glslink{JWT}{JSON Web Token}}]
    {jwt}{JWT}{JSON Web Token}

\newacronym{ssms}{SSMS}{SQL Server Management Studio}

\newacronym{ui}{UI}{User Interface (Interfaccia Utente)}

% Glossary entries
\newglossaryentry{API} {
    name={Application Program Interface},
    text=API,
    sort=api,
    description={In italiano, interfaccia di programmazione di un'applicazione. Insieme di regole e di protocolli che permette la comunicazione e lo scambio di dati tra più \textit{software} o parti di \textit{software}}
}

\newglossaryentry{code-behind}{
    name=Code-behind,
    text=code-behind,
    sort=code-behind,
    description={Pratica di porre in due \textit{file} separati il codice che gestisce la struttura e presentazione di un'applicazione e quello che ne gestisce il comportamento. È una pratica utilizzata principalmente nella programmazione in ambienti Microsoft dove il codice di struttura e presentazione è definito nei \textit{file} \texttt{.xaml}, mentre il comportamento è definito nei \textit{file} \texttt{.cs}}
}

\newglossaryentry{mockup}{
    name=Mockup,
    text=mockup,
    sort=mockup,
    description={Modello utilizzato per mostrare il \textit{design} di un prodotto. In informatica è il risultato della progettazione di un'interfaccia grafica, utilizzato come riferimento per la sua implementazione}
}

\newglossaryentry{codebase}{
    name=Codebase,
    text=codebase,
    sort=codebase,
    description={Insieme del codice sorgente che sta alla base di un \textit{software}}
}

\newglossaryentry{libreria}{
    name=Libreria,
    text=librerie,
    sort=libreria,
    description={Insieme di risorse in sola lettura che possono essere incluse e utilizzate all'interno di un \textit{software}}
}

\newglossaryentry{IDE}{
    name={Integrated Development Environment},
    text=IDE,
    sort=ide,
    description={In italiano, ambiente di sviluppo integrato. Applicazione utilizzata per lo sviluppo \textit{software} che offre, oltre ad un \textit{editor} di testo, diverse funzionalità che aiutano il programmatore durante la scrittura del codice, ad esempio sistemi di automazione e \textit{debugging}}
}

\newglossaryentry{hotreload}{
    name={Hot reload},
    text={hot reload},
    sort=hotreload,
    description={Funzionalità presente in alcuni \acrog{ide} che permette di applicare le modifiche effettuate ad un'applicazione durante la sua esecuzione, senza dover quindi chiudere l'app e rieseguirla}
}

\newglossaryentry{JWT}{
    name={JSON Web Token},
    text=JWT,
    sort=jwt,
    description={Standard \textit{web} per lo scambio di dati basato su un oggetto chiamato \textit{token} suddiviso in tre parti: \textit{header}, che contiene le informazioni che identificano l'algoritmo di codifica utilizzato, \textit{payload}, che contiene i dati codificati e in formato JSON, e una \textit{signature}, ovvero una firma che dà validità al \textit{token}}
  }

  \newglossaryentry{garbagecollector}{
    name={Garbage collector},
    text={garbage collector},
    sort=garbagecollector,
    description={Sistema di gestione automatica della memoria allocata da un programma in esecuzione. Rileva i blocchi di memoria non più utilizzati e li libera per migliorare le \textit{performance} del programma che sta eseguendo}
  }
